\documentclass{article}
\usepackage[T1]{fontenc}
\usepackage{titlesec}
\usepackage{listings}
\usepackage{color}
\usepackage[a4paper, total={6in, 10in}]{geometry}
\usepackage{amsmath}
\usepackage[document]{ragged2e}
\usepackage[ngerman]{babel}

\definecolor{my_red}{RGB}{216,60,104}
\definecolor{my_green}{RGB}{68,189,77}
\definecolor{my_grey}{RGB}{78,90,107}
\titleformat{\section}{\large\bfseries}{}{0em}{}
\titleformat{\subsection}{\bfseries}{}{0em}{}
\titleformat{\subsubsection}{}{}{2em}{}

\lstdefinestyle{customCStyle}{
  language=C,
  numbers=left,
  stepnumber=1,
  breaklines=true,
  showstringspaces=false,
  keywordstyle=\color{my_red},
  stringstyle=\color{my_green},
  commentstyle=\color{my_grey},
  morecomment=[l][\color{magenta}]
}
\lstset{basicstyle=\ttfamily\small,style=customCStyle}

\begin{document}
\title{Aufgabenblatt 11: Dynamische Datenstrukturen}
\author{Florian Ludewig (Übungsgruppe 2)}
\maketitle
\section{Aufgabe 1 -- Einfach verkettete Listen}
Ich habe die Aufgabe so interpretiert, dass \lstinline{insert_sorted} die Zahlen aufsteigend ordnet.
\begin{lstlisting}
void insert_sorted(int val) {
  struct node *temp = head;
  while (temp -> next != NULL && temp -> next -> data < val) {
    temp = temp -> next;
  }
  struct node *inserted = makeNode(val);
  inserted -> next = temp -> next;
  temp -> next = inserted;
}

struct node *reverse() {
  struct node *previous = NULL;
  struct node *current = head;
  struct node *next = NULL;
  while (current != NULL) {
    next = current -> next;
    current -> next = previous;
    previous = current;
    current = next;
  }
  return previous;
}
\end{lstlisting}
\section{Aufgabe 2 -- Stapel}
\begin{lstlisting}
  // ...
\end{lstlisting}
\end{document}