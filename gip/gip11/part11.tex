\documentclass{article}
\usepackage[T1]{fontenc}
\usepackage{titlesec}
\usepackage{listings}
\usepackage{color}
\usepackage[a4paper, total={6in, 10in}]{geometry}
\usepackage{amsmath}
\usepackage[document]{ragged2e}
\usepackage[ngerman]{babel}

\definecolor{my_red}{RGB}{216,60,104}
\definecolor{my_green}{RGB}{68,189,77}
\definecolor{my_grey}{RGB}{78,90,107}
\titleformat{\section}{\large\bfseries}{}{0em}{}
\titleformat{\subsection}{\bfseries}{}{0em}{}
\titleformat{\subsubsection}{}{}{2em}{}

\lstdefinestyle{customCStyle}{
  language=C,
  numbers=left,
  stepnumber=1,
  breaklines=true,
  showstringspaces=false,
  keywordstyle=\color{my_red},
  stringstyle=\color{my_green},
  commentstyle=\color{my_grey},
  morecomment=[l][\color{magenta}]
}
\lstset{basicstyle=\ttfamily\small,style=customCStyle}

\begin{document}
\title{Aufgabenblatt 11: Dynamische Datenstrukturen}
\author{Florian Ludewig (Übungsgruppe 2)}
\maketitle
\section{Aufgabe 1 -- Einfach verkettete Listen}
Ich habe die Aufgabe so interpretiert, dass \lstinline{insert_sorted} die Zahlen aufsteigend ordnet.
\begin{lstlisting}
void insert_sorted(int val) {
  struct node *temp = head;
  while (temp -> next != NULL && temp -> next -> data < val) {
    temp = temp -> next;
  }
  struct node *inserted = makeNode(val);
  inserted -> next = temp -> next;
  temp -> next = inserted;
}

struct node *reverse() {
  struct node *previous = NULL;
  struct node *current = head;
  struct node *next = NULL;
  while (current != NULL) {
    next = current -> next;
    current -> next = previous;
    previous = current;
    current = next;
  }
  return previous;
}
\end{lstlisting}
\section{Aufgabe 2 -- Stapel}
\begin{lstlisting}
#include<stdio.h>
#include <stdlib.h>
#include <string.h>

struct node {
  char data;
  struct node *next;
};

struct node *top;

struct node *makeNode(char val) {
  struct node *node = NULL;
  node = malloc(sizeof(struct node));
  if (node != NULL) {
    node -> data = val;
    node -> next = NULL;
    return node;
  } else {
    return NULL;
  }
}

void push(char c){
  struct node *p = makeNode(c);
  p -> next = top;
  top = p;
}

int pop(void){
  int result = top -> data;
  struct node *p = top;
  top = top -> next;
  free(p);
  return result;
}

int is_opening_bracket(char c) { return c == '{' || c == '[' || c == '('; }
int is_closing_bracket(char c) { return c == ')' || c == ']' || c == '}'; }
int is_matching_bracket(char opening, char closing) {
  if (opening == '(') return closing == ')';
  if (opening == '[') return closing == ']';
  if (opening == '{') return closing == '}';
  return 0;
}

void print_error(char input[], int position) {
    printf("Fehlerhaft Klammerung:\n");
    printf("%s\n", input);
    for (int i = 0; i < position; i++) {
      printf(" ");
    }
    printf("^\n");
}

void validate_brackets(char input[]) {
  size_t length = strlen(input);
  for (int i = 0; i < length; i++) {
    char c = input[i];
    if (is_opening_bracket(c)) {
      push(c);
    }
    if (is_closing_bracket(c)) {
      if (is_matching_bracket(top -> data, c)) {
        pop();
      } else {
        print_error(input, i);
        return;
      }
    }
  }
  if (!top) {
    printf("Korrekte Klammerung\n");
  } else {
    print_error(input, length);
  }
}

int main(void) {
  char input[256];
  printf("Bitte Ausdruck eingeben: ");
  scanf("%s", &input[0]);
  validate_brackets(input);
  return 1;
}
\end{lstlisting}
\end{document}