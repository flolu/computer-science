\documentclass{article}
\usepackage[T1]{fontenc}
\usepackage{titlesec}
\usepackage{listings}
\usepackage{color}
\usepackage[a4paper, total={6in, 10in}]{geometry}
\usepackage{amsmath}

\titleformat{\section}{\large\bfseries}{}{0em}{}
\titleformat{\subsection}{\bfseries}{}{0em}{}
\titleformat{\subsubsection}{}{}{2em}{}

\lstdefinestyle{customCStyle}{
  language=C,
  numbers=left,
  stepnumber=1,
  breaklines=true,
  showstringspaces=false,
  keywordstyle=\color{blue},
  stringstyle=\color{red},
  commentstyle=\color{green},
  morecomment=[l][\color{magenta}]
}
\lstdefinestyle{customCStyle2}{language=C,breaklines=true}
\lstset{basicstyle=\ttfamily\small,style=customCStyle2}

\begin{document}
\title{Aufgabenblatt 3}
\author{Florian Ludewig (Übungsgruppe 2)}
\maketitle
\section{Aufgabe 1}
\subsection{a)}


\begin{tabular}{ | l | l | }

\hline Ausdruck & Wert \\ \hline

\begin{lstlisting}
int x = 49;
\end{lstlisting} & -- \\ \hline

\begin{lstlisting}
!(x >= 0 && x < 50 || x > 50 && x <= 100)
\end{lstlisting} &
\begin{lstlisting}
false
\end{lstlisting} \\ \hline

\begin{lstlisting}
x++ == 49 || x-- == 49
\end{lstlisting} &
\begin{lstlisting}
true
\end{lstlisting} \\ \hline

\begin{lstlisting}
!(x >= 0 && x < 50 || x > 50 && x <= 100)
\end{lstlisting} &
\begin{lstlisting}
true
\end{lstlisting} \\ \hline

\begin{lstlisting}
int k; scanf("%d", &k);
\end{lstlisting} & -- \\ \hline

\begin{lstlisting}
!(k > 0 && k%10 <= 7) == (!(k > 0) || !(k%10 <= 7))
\end{lstlisting} \\ \\
\( \neg(k>0 \land k(mod \; 10) \leq 7)=(\neg(k>0)\lor \neg(k(mod \; 10)\leq 7)) \)\\
Definiere \(P:=k>0\) und \(Q:=k(mod \; 10)\leq 7\).\\
Dann gilt: \(\neg(P \land Q) \Leftrightarrow \neg P \lor \neg Q\), weil es laut\\den De Morganschen Gesetzen äquivalent ist. &
\begin{lstlisting}
true
\end{lstlisting} \\ \hline
\end{tabular}


\subsection{b)}
\begin{lstlisting}
1) !(m > 10) && (n != 0)
\end{lstlisting}
\begin{lstlisting}
2) (x > 0) && (y > 0) && (z > 0) && (x % 5 == 0) && (y % 5 == 0) && (z % 5 == 0)
\end{lstlisting}
\begin{lstlisting}
3) (a >= 1 && a <= 10) || (a < -7)
\end{lstlisting}
\begin{lstlisting}
4) 1 < 3
\end{lstlisting}

\section{Aufgabe 2}
\lstset{basicstyle=\ttfamily\small,style=customCStyle}
\begin{lstlisting}
#include <stdio.h>
#include <stdbool.h>
#include <stdlib.h>

struct station
{
  int line;
  int stop;
};

struct trip
{
  struct station start;
  struct station end;
};

bool compare_stations(struct trip t, struct station s1, struct station s2)
{
  return (t.start.line == s1.line && t.start.stop == s1.stop && t.end.line == s2.line && t.end.stop == s2.stop) ||
          (t.start.line == s2.line && t.start.stop == s2.stop && t.end.line == s1.line && t.end.stop == s1.stop);
}

bool is_short_trip(struct trip t)
{
  if (compare_stations(t, (struct station){5, 2}, (struct station){4, 2}) ||
      compare_stations(t, (struct station){0, 0}, (struct station){2, 1}))
    return false;

  if (abs(t.start.stop - t.end.stop) == 1 &&
      (t.start.line == t.end.line ||
        t.start.line == 0 || t.end.line == 0))
    return true;

  if (abs(t.start.line % 5 - t.end.line % 5) == 1 &&
      t.start.stop == 2 && t.end.stop == 2)
    return true;

  return false;
}

int count_zone_crosses(struct trip t)
{
  if (t.start.stop > 2 && t.end.stop > 2 && t.start.line != t.end.line)
    return 2;
  if ((t.start.stop > 2 && t.end.stop < 3) || (t.start.stop < 3 && t.end.stop > 2))
    return 1;

  return 0;
}

int count_end_stops(struct trip t)
{
  return (t.start.stop == 5 ? 1 : 0) + (t.end.stop == 5 ? 1 : 0);
}

bool is_start_equal_to_end(struct trip t)
{
  return t.start.line == t.end.line && t.start.stop == t.end.stop;
}

int main()
{
  int start_input, end_input;
  printf("Starthaltestelle: ");
  scanf("%d", &start_input);
  printf("Zielhaltestelle: ");
  scanf("%d", &end_input);

  struct trip trip = {
      {(start_input / 10) % 10, start_input % 10},
      {(end_input / 10) % 10, end_input % 10}};

  int price = 0;
  price = is_short_trip(trip) ? 2 : 3;
  price += count_zone_crosses(trip);
  price += count_end_stops(trip);
  price = is_start_equal_to_end(trip) ? 0 : price;

  printf("%d Euro\n", price);
  return 0;
}
\end{lstlisting}
\end{document}