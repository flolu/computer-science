\documentclass{article}
\usepackage[T1]{fontenc}
\usepackage{titlesec}
\usepackage{listings}
\usepackage{color}
\usepackage[a4paper, total={6in, 10in}]{geometry}
\usepackage{amsmath}
\usepackage[document]{ragged2e}
\usepackage[ngerman]{babel}

\definecolor{my_red}{RGB}{216,60,104}
\definecolor{my_green}{RGB}{68,189,77}
\definecolor{my_grey}{RGB}{78,90,107}
\titleformat{\section}{\large\bfseries}{}{0em}{}
\titleformat{\subsection}{\bfseries}{}{0em}{}
\titleformat{\subsubsection}{}{}{2em}{}

\lstdefinestyle{customCStyle}{
  language=C,
  numbers=left,
  stepnumber=1,
  breaklines=true,
  showstringspaces=false,
  keywordstyle=\color{my_red},
  stringstyle=\color{my_green},
  commentstyle=\color{my_grey},
  morecomment=[l][\color{magenta}]
}
\lstset{basicstyle=\ttfamily\small,style=customCStyle}

\begin{document}
\title{Aufgabenblatt 9: Reihungen (2)}
\author{Florian Ludewig (Übungsgruppe 2)}
\maketitle
\section{Aufgabe 1 -- Binäre Suche}
\subsection{a)}
\begin{lstlisting}
int linearSearch(int n, int a[], int l) {
  for (int i = 0; i < l; i++) {
    if (a[i] == n) return i;
  }
  return -1;
}

int binarySearch(int n, int a[], int l) {
  int range_start = 0, range_end = l - 1;
  while (1) {
    int middle = (range_start + range_end) / 2;
    if (a[middle] == n) return middle;
    if (range_start > range_end) return -1;
    if (a[middle] < n) range_start = middle + 1;
    else range_end = middle - 1;
  }
}
\end{lstlisting}
\subsection{b)}
\begin{lstlisting}
int main(void) {
  int n;
  printf("Zahl eingeben: ");
  scanf("%d", &n);

  int primes[25] = {2, 3, 5, 7, 11, 13, 17, 19, 23, 29, 31, 37, 41, 43, 47, 53, 59, 61, 67, 71, 73, 79, 83, 89, 97};
  int position_linear_search = linearSearch(n, primes, 25);
  int position_binary_search = binarySearch(n, primes, 25);

  printf("Suchergebnis der linearen Suche: %d\n", position_linear_search);
  printf("Suchergebnis der binaeren Suche: %d\n", position_binary_search);

  return 0;
}
\end{lstlisting}
\subsection{c)}
\begin{lstlisting}
int binarySearch(int n, int a[], int l) {
  int* range_start = &a[0];
  int* range_end = &a[l - 1];
  while(1) {
    if (range_start > range_end) return -1;
    int* middle = ((range_end - range_start) / 2) + range_start;
    if (*middle == n) return middle - &a[0];
    if (*middle > n) range_end = middle - 1;
    else range_start = middle + 1;
  }
}
\end{lstlisting}
\pagebreak
\section{Aufgabe 2 -- Magisches Quadrat}
\begin{lstlisting}
#include<stdio.h>

void generate_magic_square(int n) {
  int square[n][n];

  for (int i = 0; i < n; i++)
    for (int j = 0; j < n; j++)
      square[i][j] = 0;

  int row = n - 1, column = n / 2;
  int successor = 2;
  square[row][column] = 1;

  while(successor <= n * n) {
    row = row == n - 1 ? 0 : row + 1;
    column = column == n - 1 ? 0 : column + 1;
    if (square[row][column] == 0) {
      square[row][column] = successor;
      successor++;
    } else {
      while (1) {
        row = row == n - 1 ? 0 : row + 1;
        column = column == 0 ? n - 1 : column - 1;
        if (square[row][column] == 0) {
          square[row][column] = successor;
          successor++;
          break;
        }
      }
    }
  }

  for (int i = 0; i < n; i++) {
    for (int j = 0; j < n; j++)
      printf(" %d ", square[i][j]);
    printf("\n");
  }
  printf("\n");
}

int main(void) {
  int n;
  printf("Ungerade Zahl eingeben: ");
  scanf("%d", &n);
  generate_magic_square(n);
  return 0;
}
\end{lstlisting}
\end{document}