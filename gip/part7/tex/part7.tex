\documentclass{article}
\usepackage[T1]{fontenc}
\usepackage{titlesec}
\usepackage{listings}
\usepackage{color}
\usepackage[a4paper, total={6in, 10in}]{geometry}
\usepackage{amsmath}
\usepackage[document]{ragged2e}

\definecolor{my_red}{RGB}{216,60,104}
\definecolor{my_green}{RGB}{68,189,77}
\definecolor{my_grey}{RGB}{78,90,107}
\titleformat{\section}{\large\bfseries}{}{0em}{}
\titleformat{\subsection}{\bfseries}{}{0em}{}
\titleformat{\subsubsection}{}{}{2em}{}

\lstdefinestyle{customCStyle}{
  language=C,
  numbers=left,
  stepnumber=1,
  breaklines=true,
  showstringspaces=false,
  keywordstyle=\color{my_red},
  stringstyle=\color{my_green},
  commentstyle=\color{my_grey},
  morecomment=[l][\color{magenta}]
}
\lstset{basicstyle=\ttfamily\small,style=customCStyle}

\begin{document}
\title{Aufgabenblatt 7: Zeiger und Funktionen}
\author{Florian Ludewig (Übungsgruppe 2)}
\maketitle
\section{Aufgabe 1 -- Zeiger in C}
\begin{lstlisting}
#include <stdio.h>

int main()
{
  int var = 10;
  int *p;
  p = &var;

  printf("Ausgabe 1: %p \n", &var);
  printf("Ausgabe 2: %d \n", *p);
  printf("Ausgabe 3: %d \n", *(&var));
  printf("Ausgabe 4: %p \n", &p);

  return 0;
}
\end{lstlisting}
\subsection{Ausgabe 1}
\lstinline{printf("Ausgabe  1: %p \n", &var);} liefert foldende Ausgabe: \lstinline{Ausgabe 1: 0x7fff086e611c}. Durch das \lstinline{&} Symbol wird die Speicheradresse der nachstehenden Variable, in diesem Fall von \lstinline{var}, ausgegeben.
\linebreak
\subsection{Ausgabe 2}
\lstinline{printf("Ausgabe 2: %d \n", *p);} gibt \lstinline{Ausgabe 2: 10} aus. Durch \lstinline{p = &var;} wurde in Zeile 7 dem Zeiger \lstinline{p} die Speicheradresse von dem Integer \lstinline{var} zugewiesen. Nun wird mithilfe des \lstinline{*} Symbols der Wert ausgelesen der in der Speicheradresse von dem Zeiger \lstinline{p} gepspeichert wird (in dem Fall der Wert der Variable \lstinline{var}, also \lstinline{10}).
\linebreak
\subsection{Ausgabe 3}
\lstinline{printf("Ausgabe 3: %d \n", *(&var));} liefert foldende Ausgabe: \lstinline{Ausgabe 3: 10}. Hier gibt \lstinline{&var} die Speicheradresse von der Variable \lstinline{var} zurück. Durch das \lstinline{*} Symbol wird dann der Wert an dieser Speicheradresse ausgelesen. Dieser Wert ist \lstinline{10}, weil in Zeile \lstinline{5} dem Integer \lstinline{var} der Wert \lstinline{10} zugewiesen wurde.
\linebreak
\subsection{Ausgabe 4}
\lstinline{printf("Ausgabe 4: %p \n", &p);} gibt \lstinline{Ausgabe 4: 0x7fff086e6120} aus. Durch das \lstinline{&} Symbol wird die Speicheradresse von dem Pointer \lstinline{p} ausgegeben. Es handelt sich hierbei also nicht um die Adresse, die in dem Pointer \lstinline{p} gepspeichert ist, sondern um die Adresse an welcher der Pointer \lstinline{p} selbst gepspeichert ist. Aus diesem Grund ist Ausgabe 4 auch verschieden von Ausgabe 1.
\pagebreak
\section{Aufgabe 2 -- Wertetabellen}
\subsection{a)}
\begin{lstlisting}
#include<stdio.h>

double quadrat(double x) { return x*x; }

void wertetabelle(double (*f)(double), double min, double max, double step) {
  int steps = (max - min) / step;
  for (int i = 0; i <= steps; i++) {
    double x = min + (i * step);
    printf("f(%lf) = %lf\n", x, f(x));
  }
}

int main() {
  wertetabelle(quadrat, 0.0, 10.0, 1.0);
  return 0;
}
\end{lstlisting}
\subsection{b)}
\begin{lstlisting}
#include<stdio.h>
#include <stdarg.h>

double cubic(double x) { return x*x*x+2.0; }

void wertetabelle(double (*f)(double), int count, ...) {
  va_list args;
  va_start(args, count);
  for (int i = 0; i < count; i++) {
    double x = va_arg(args, double);
    printf("f(%lf) = %lf\n", x, f(x));
  }
  va_end(args);
}

int main() {
  wertetabelle(cubic, 4, 3.0, 7.0, 11.0, 41.0);
  return 0;
}
\end{lstlisting}
\end{document}