\documentclass{article}
\usepackage{hyperref}
\begin{document}

  \begin{center}
     \Large\textbf{Aufgabenblatt 2}\\
     \large{Florian Ludewig, Übungsgruppe 2}
  \end{center}

  \bf{Aufgabe 1}\\
  \textnormal{a)}
  \begin{verbatim}
#include <stdio.h>

int main(void)
{
  int n;
  printf("Geben Sie eine positive ganze Zahl ein: ");
  scanf("%d", &n);
  printf("Die Fakultät von %d ist: ", n);
  int f = 1;
  while (n > 0)
  {
    f = f * n;
    n = n - 1;
  }
  printf("%d\n", f);
  return 0;
}
    \end{verbatim}

  \textnormal{b)}
  \begin{verbatim}
#include <stdio.h>

int main(void)
{
  int n, i, q;
  printf("Geben Sie eine ganze Zahl ein: ");
  scanf("%d", &n);
  q = 0;
  for (i = n; i > 0; i = i / 10)
  {
    q += i % 10;
  }
  printf("Die Quersumme von %d ist: %d\n", n, q);
  return 0;
}
    \end{verbatim}

  \bf{Aufgabe 2}\\
  \bf{Aufgabe 3}\\
\end{document}