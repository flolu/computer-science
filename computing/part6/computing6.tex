\documentclass{article}
\usepackage[utf8]{inputenc}
\usepackage[german]{babel}
\usepackage{listings}
\usepackage{color}
\usepackage{graphicx}
\usepackage[a4paper, total={6in, 10in}]{geometry}

\title{Rechnerstrukturen: Übungsblatt 6}
\author{Florian Ludewig (185722)}
\date\today

\definecolor{my_red}{RGB}{216,60,104}
\definecolor{my_green}{RGB}{68,189,77}
\definecolor{my_grey}{RGB}{78,90,107}
\lstdefinestyle{customCStyle}{
  language=C,
  numbers=left,
  stepnumber=1,
  breaklines=true,
  showstringspaces=false,
  keywordstyle=\color{my_red},
  stringstyle=\color{my_green},
  commentstyle=\color{my_grey},
  morecomment=[l][\color{magenta}]
}
\lstset{basicstyle=\ttfamily\small,style=customCStyle}

\begin{document}
\maketitle

\section*{Aufgabe 2}
\includegraphics[width=\textwidth]{computing6_2.png}

\pagebreak
\section*{Aufgabe 3}
\begin{lstlisting}
unsigned i2f(int i) {
  if (i == 0) return 0;
  int abs = i < 0 ? -i : i;

  int s = i > 0 ? 0b0 : 0x80000000;

  int tmp = abs;
  int right_shifts = 0;
  while (0b01 != abs) {
    abs = abs >> 1;
    right_shifts++;
  }
  int e = (127 + right_shifts) << 23;

  tmp = (tmp << (32 - right_shifts)) >> 9;
  int m = tmp & 0x007FFFFF;

  return s | e | m;
}
\end{lstlisting}

\end{document}