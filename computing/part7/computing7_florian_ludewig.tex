\documentclass{article}
\usepackage[utf8]{inputenc}
\usepackage[german]{babel}
\usepackage{listings}
\usepackage{color}
\usepackage{graphicx}
\usepackage[a4paper, total={6in, 10in}]{geometry}

\title{Rechnerstrukturen: Übungsblatt 7}
\author{Florian Ludewig (185722)}

\definecolor{my_red}{RGB}{216,60,104}
\definecolor{my_green}{RGB}{68,189,77}
\definecolor{my_grey}{RGB}{78,90,107}
\lstdefinestyle{customCStyle}{
  language=C,
  numbers=left,
  stepnumber=1,
  breaklines=true,
  showstringspaces=false,
  keywordstyle=\color{my_red},
  stringstyle=\color{my_green},
  commentstyle=\color{my_grey},
  morecomment=[l][\color{magenta}]
}
\lstset{basicstyle=\ttfamily\small,style=customCStyle}

\begin{document}
\maketitle

\section*{Aufgabe 3}

\begin{itemize}
	\item 1 - \texttt{moveb} verschiebt nur 8 Bit, aber \texttt{\%ebx} ist 64 Bit groß
	\item 2 - \texttt{movel} verschiebt nur 32 Bit, aber wir wollen 64 Bit verschieben \texttt{\%rsp}
	\item 3 - /
	\item 4 - \texttt{\%sl} existiert nicht, gemeint ist wahrscheinlich \texttt{\%sil}
	\item 5 - \texttt{0x123} hat 12 Bit, es würde auch \texttt{movew} genügen
	\item 6 - /
	\item 7 - /
	\item 8 - /
\end{itemize}

\section*{Aufgabe 4}
\begin{lstlisting}
movl %eax, (%rsp)
movw (%rax), %dx
movb $0xFF, %bl
\end{lstlisting}

\end{document}