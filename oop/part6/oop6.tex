\documentclass{article}
\usepackage[utf8]{inputenc}
\usepackage[german]{babel}
\usepackage[a4paper, total={6in, 10in}]{geometry}
\usepackage{listings}
\usepackage{color}

\title{Objektorientierte Programmierung: Aufgabenblatt 6}
\author{Florian Ludewig (185722)}
\date\today


\definecolor{my_red}{RGB}{216,60,104}
\definecolor{my_green}{RGB}{68,189,77}
\definecolor{my_grey}{RGB}{78,90,107}
\lstdefinestyle{customCStyle}{
  language=Java,
  numbers=left,
  stepnumber=1,
  breaklines=true,
  showstringspaces=false,
  keywordstyle=\color{my_red},
  stringstyle=\color{my_green},
  commentstyle=\color{my_grey},
  morecomment=[l][\color{magenta}]
}
\lstset{basicstyle=\ttfamily\small,style=customCStyle}

\begin{document}
\maketitle

\section*{Aufgabe 1}
\texttt{App.java}
\begin{lstlisting}
public class App {
  public static void main(String[] args) throws Exception {
    // contained
    new Rectangle(2.5, 2.5, 4, 4).getLocationRelation(new Rectangle(1, 2.5, 1, 1));

    // aligned
    new Rectangle(0, 0.5, 3, 2).getLocationRelation(new Rectangle(2, -1, 2, 2));

    // same
    new Rectangle(2, -2.1, 2, 2).getLocationRelation(new Rectangle(2, -2.1, 2, 2));

    // disjoint
    new Rectangle(0, 0, 1, 1).getLocationRelation(new Rectangle(-100, 100, 1, 1));
  }
}
\end{lstlisting}
\texttt{Figure.java}
\begin{lstlisting}
abstract class Figure {
  protected double x;
  protected double y;

  Figure() {
    x = 0;
    y = 0;
  }

  Figure(double x, double y) {
    this.x = x;
    this.y = y;
  }

  public double getX() {
    return x;
  }

  public double getY() {
    return y;
  }

  void printXY() {
    System.out.println("X: " + getX() + " Y: " + getY());
  }
}
\end{lstlisting}
\texttt{MobileObject.java}
\begin{lstlisting}
interface MobileObject {
  void move(double x, double y);

  void increase(double value);

  void decrease(double value);
}
\end{lstlisting}
\texttt{Rectangle.java}
\begin{lstlisting}
public class Rectangle extends Figure implements MobileObject {
  private double height;
  private double width;

  Rectangle(double h, double w) {
    super();
    height = h;
    width = w;
  }

  Rectangle(double x, double y, double h, double w) {
    super(x, y);
    height = h;
    width = w;
  }

  double getHeight() {
    return height;
  }

  double getWidth() {
    return width;
  }

  void printXY() {
    super.printXY();
    System.out.println("Height: " + getHeight() + " Width: " + getWidth());
  }

  public void move(double x, double y) {
    this.x = x;
    this.y = y;
  }

  public void increase(double value) {
    width *= value;
    height *= value;
  }

  public void decrease(double value) {
    width *= 1 / value;
    height *= 1 / value;
  }

  void getLocationRelation(Rectangle r) {
    double intersectionY = intersectionYAxis(r);
    double intersectionX = intersectionXAxis(r);

    if (intersectionY < 0 || intersectionX < 0) {
      System.out.println("disjoint");
      return;
    }

    if (intersectionY == r.getHeight() && intersectionY == getHeight() && intersectionX == r.getWidth()
        && intersectionX == getWidth()) {
      System.out.println("same");
      return;
    }

    if (intersectionY == Math.min(r.getHeight(), getHeight()) && intersectionX == Math.min(r.getWidth(), getWidth())) {
      System.out.println("contained");
      return;
    }

    if ((intersectionX == 0 && intersectionY != 0) || (intersectionX != 0 && intersectionY == 0)) {
      System.out.println("aligned");
      return;
    }

    System.out.println("something else");
  }

  private double intersectionYAxis(Rectangle r) {
    return getIntersection(y - getHeight() / 2, y + getHeight() / 2, r.y - r.getHeight() / 2, r.y + r.getHeight() / 2);
  }

  private double intersectionXAxis(Rectangle r) {
    return getIntersection(x - getWidth() / 2, x + getWidth() / 2, r.x - r.getWidth() / 2, r.x + r.getWidth() / 2);
  }

  private double getIntersection(double min1, double max1, double min2, double max2) {
    if (max2 > max1 && min1 > min2)
      return max1 - min1;
    if (max1 > max2 && min2 > min1)
      return max2 - min2;
    return max1 > max2 ? max2 - min1 : max1 - min2;
  }
}
\end{lstlisting}
\section*{Aufgabe 2}
\texttt{Roman.java}
\begin{lstlisting}
public class Roman {
  private String romanString;
  private int romanValue;

  Roman(String roman) {
    this.romanString = roman;
    romanValue = romanToInt(roman);
  }

  Roman(int value) {
    romanValue = value;
    romanString = intToRoman(value);
  }

  public Roman add(Roman r) {
    return new Roman(getValue() + r.getValue());
  }

  public Roman substract(Roman r) {
    int result = getValue() - r.getValue();
    if (result <= 0)
      throw new Error("negative numbers or zero no allowed");
    return new Roman(result);
  }

  public Roman multiply(Roman r) {
    return new Roman(getValue() * r.getValue());
  }

  public Roman divide(Roman r) {
    return new Roman(getValue() / r.getValue());
  }

  public String toString() {
    return romanString;
  }

  public int hashCode() {
    return getValue();
  }

  public int getValue() {
    return romanValue;
  }

  public boolean equals(Roman r) {
    return r.hashCode() == hashCode();
  }

  private String intToRoman(int value) {
    String roman = "";
    String[] chars = new String[] { "I", "IV", "V", "IX", "X", "XL", "L", "XC", "C", "CD", "D", "CM", "M" };
    int[] values = new int[] { 1, 4, 5, 9, 10, 40, 50, 90, 100, 400, 500, 900, 1000 };

    int counter = 0;
    for (int i = values.length - 1; i >= 0; i--) {
      counter = value / values[i];
      value = value % values[i];
      for (int j = 0; j < counter; j++)
        roman += chars[i];
    }
    return roman;
  }

  private int romanToInt(String roman) {
    int value = 0;
    String[] binaries = new String[] { "IV", "IX", "XL", "XC", "CD", "CM" };

    while (roman.length() > 0) {
      if (roman.length() > 1 && stringIncludes(roman.substring(0, 2), binaries)) {
        value += romanCharToValue(roman.substring(0, 2));
        roman = roman.substring(2, roman.length());
      } else {
        value += romanCharToValue(roman.substring(0, 1));
        roman = roman.substring(1, roman.length());
      }
    }

    return value;
  }

  private int romanCharToValue(String romanChar) {
    String[] chars = new String[] { "I", "IV", "V", "IX", "X", "XL", "L", "XC", "C", "CD", "D", "CM", "M" };
    int[] values = new int[] { 1, 4, 5, 9, 10, 40, 50, 90, 100, 400, 500, 900, 1000 };
    for (int i = 0; i < chars.length; i++) {
      if (romanChar.equals(chars[i]))
        return values[i];
    }
    return 0;
  }

  private boolean stringIncludes(String search, String[] strings) {
    for (String s : strings) {
      if (s.equals(search))
        return true;
    }
    return false;
  }
}
\end{lstlisting}

\texttt{App.java}
\begin{lstlisting}
public class App {
  public static void main(String[] args) {
    System.out.println(new Roman(4).equals(new Roman("IV"))); // true
    System.out.println(new Roman(72).equals(new Roman("LXXII"))); // true
    System.out.println(new Roman("DCCCXLV").equals(new Roman(845))); // true
    System.out.println(new Roman("MMXXII").equals(new Roman(2022))); // true

    System.out.println(new Roman("XIV").equals(new Roman(104))); // false
    System.out.println(new Roman(104).equals(new Roman("XIV"))); // false
  }
}
\end{lstlisting}


\end{document}