\documentclass{article}
\usepackage[utf8]{inputenc}
\usepackage[german]{babel}
\usepackage[a4paper, total={6in, 10in}]{geometry}
\usepackage{listings}
\usepackage{color}

\title{Objektorientierte Programmierung: Aufgabenblatt 8}
\author{Florian Ludewig (185722)}
\date\today


\definecolor{my_red}{RGB}{216,60,104}
\definecolor{my_green}{RGB}{68,189,77}
\definecolor{my_grey}{RGB}{78,90,107}
\lstdefinestyle{customCStyle}{
  language=Java,
  numbers=left,
  stepnumber=1,
  breaklines=true,
  showstringspaces=false,
  keywordstyle=\color{my_red},
  stringstyle=\color{my_green},
  commentstyle=\color{my_grey},
  morecomment=[l][\color{magenta}]
}
\lstset{basicstyle=\ttfamily\small,style=customCStyle}

\begin{document}
\maketitle

\section*{Aufgabe 1}
\texttt{Node.java}
\begin{lstlisting}
public class Node {
  public int value;
  public Node next;

  public Node(int val, Node next) {
    value = val;
    this.next = next;
  }

  protected Node clone() {
    if (next == null)
      return new Node(value, null);
    return new Node(value, next.clone());
  }
}
\end{lstlisting}
\texttt{SimpleList.java}
\begin{lstlisting}
public class SimpleList {
  private Node head = null;

  SimpleList() {
  }

  SimpleList(Node initialHead) {
    head = initialHead;
  }

  public void add(int i) {
    head = new Node(i, head);
  }

  protected SimpleList clone() {
    return new SimpleList(head.clone());
  }
}
\end{lstlisting}

\section*{Aufgabe 2}
\subsection*{1)}
\texttt{ActionObject.java}
\begin{lstlisting}
public interface ActionObject {
  void action(Node n);
}
\end{lstlisting}
\texttt{CountStringAction1.java}
\begin{lstlisting}
public class CountStringAction1 implements ActionObject {
  private int counter = 0;
  private String search;

  CountStringAction1(String str) {
    search = str;
  }

  public void action(Node n) {
    if (n.data instanceof String && n.data == search)
      counter++;
  }

  public int getCounter() {
    return counter;
  }
}
\end{lstlisting}
\texttt{List.java}
\begin{lstlisting}
public class List {
  private Node head = null;

  void add(Object obj) {
    this.head = new Node(obj, head);
  }

  public void traverseAndApply(ActionObject p) {
    for (Node cursor = head; cursor != null; cursor = cursor.next) {
      p.action(cursor);
    }
  }

  void print() {
    for (Node cursor = head; cursor != null; cursor = cursor.next) {
      System.out.println(cursor.data.toString());
    }
  }
}
\end{lstlisting}
\texttt{Node.java}
\begin{lstlisting}
public class Node {
  public Object data;
  public Node next;

  Node(Object obj, Node node) {
    data = obj;
    next = node;
  }
}
\end{lstlisting}
\texttt{Main.java}
\begin{lstlisting}
public class Main {
  public static void main(String[] args) throws Exception {
    List list = new List();
    list.add("string");
    list.add(42);
    list.add("string");
    list.add("something else");
    list.add("string");
    countOccurences1(list, "string");
    countOccurences2(list, "string");
  }

  public static void countOccurences1(List list, String str) {
    CountStringAction1 cs1 = new CountStringAction1(str);
    list.traverseAndApply(cs1);
    System.out.println(str + " kommt " + cs1.getCounter() + " mal in der Liste vor");
  }

  public static void countOccurences2(List list, String str) {
    CountStringAction2 cs2 = new CountStringAction2(str);
    list.traverseAndApply(cs2);
    System.out.println(str + " kommt " + cs2.getCounter() + " mal in der Liste vor");
  }
}
\end{lstlisting}
\subsection*{2)}
\texttt{CountStringAction2.java}
\begin{lstlisting}
public class CountStringAction2 implements ActionObject {
  private int counter = 0;
  private String search;

  CountStringAction2(String str) {
    search = str;
  }

  public void action(Node n) {
    if (n.data instanceof String && n.data == search) {
      counter++;
      if (counter >= 5)
        n.data = "XXX";
    }
  }

  public int getCounter() {
    return counter;
  }
}
\end{lstlisting}

\end{document}